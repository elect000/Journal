\documentclass[a4j,twocolumn]{tarticle}
\pagestyle{empty}
\setlength{\columnseprule}{0.4pt}
\begin{document}

\begin{center}
\textbf{\Huge 汗と涙のウィンナー}
\end{center}

\section{材料}

\begin{tabular}{ll}
ウィンナー       &4、5本\\
粗挽き胡椒       &少々\\
マスタード       &大さじ\rensuji{5}\\
(あれば)くやし涙 &\rensuji{2, 3}滴\\
サラダ油         &小さじ\rensuji{1}
\end{tabular}

\section{器具}

\begin{itemize}
 \item フライパン
 \item 菜箸
 \item 大さじ
 \item 小さじ
\end{itemize}

\section{作り方}

\begin{enumerate}
 \item[イ] マスタードに、くやし涙を加えてよく練り込む。
 \item[ロ] フライパンにサラダ油をいれ、十分熱する。
 \item[ハ] ウィンナーをいれ、菜箸でよく炒める。
 \item[ニ] そのまま炒めに炒め、ウィンナーが破裂するまで炒める。
 \item[ホ] 胡椒を振って、皿にもる。
 \item[ヘ] マスタードを添えていただく。
\end{enumerate}

\section{一言}

きっと一度は彼のためにお弁当を作ったことでしょう。頑張って、汗を流して、
ウィンナーをタコの形にしてみたり、カニの形にしてみたりして。そんなにま
でしたこともあったのに、結末がこれじゃあ気持ちがおさまりませんね。はじ
けるまで炒めに炒めたウィンナーに、マスタードをたっぷりつけて、パンには
さんで食べるもよし、フォークで突き刺して食べるもよし。ちょっぴり辛子が
効きすぎて涙が出るかもしれませんが、それはきっともうくやし涙なんかでは
ないはずですよね。お好みのやりかたで、どうぞ思い切りかぶりついてくださ
い。

これで気分が少しはせいせいするかもしれませんが、それではまだあなたの傷
はふさがっていません。きっとまだささくれだっている部分もあることと思い
ます。ストレスを発散したあとは、あなたの傷を包み込んで治すお料理をどう
ぞ。
\end{document}
