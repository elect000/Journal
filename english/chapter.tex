% Intended LaTeX compiler: pdflatex
\documentclass{scrartcl}
    \usepackage{amsmath, amssymb, bm}
		\usepackage[utf8]{inputenc}
		\usepackage[dvipdfmx]{graphicx}
		\usepackage[dvipdfmx]{color}
		\usepackage[backend=biber,bibencoding=utf8]{biblatex}
		\usepackage{url}
		\usepackage{indentfirst}
		\usepackage[normalem]{ulem}
		\usepackage{longtable}
		\usepackage{minted}
		\usepackage{fancyvrb}
    \usepackage[dvipdfmx,colorlinks=false,pdfborder={0 0 0}]{hyperref}
    \usepackage{pxjahyper}
    \usepackage{caption}
		\newtheorem{th.}{Proposition}
\author{情報科学類3年 江畑 拓哉 (201611350)}
\date{}
\title{線形システムと最小二乗法}
\begin{document}

\maketitle
\tableofcontents


\section{概要}
\label{sec:org35a5a74}
\begin{align}
        \bm{A} \bm{x}  &=  \bm{b} & \text{(3-1)}\\
        &where\  \bm{A}\ \in \ \bm{R}^{n \times n} 
\end{align}
という線形問題を解くためのいくつかの定理を学びます。\\
(3-1)式は部分ピボットを用いたガウス消去法によって解くことが出来ます。これは行列を三角行列に分解することと同義です。\\
私達が考えなければならない点として、\(\bm{A}\in\bm{R}^{m \times n}\ where m > n\) の場合があり、これは最小二乗法によって解を求めます。\\
行列の分解について議論する前に、 (3-1)が単一の解を持つための条件について基本的な知識を示します。\\

\begin{th.}\\
\begin{align*}
\bm{A}\bm{x} &= \bm{b} \\ 
&where\ \bm{A} \in \bm{R}^{n \times n}\ and \ nonsingular \\ 
\end{align*}
\end{theorem}
\end{document}
